\section{Aplication}

In this section, we see a optimal Control applied to vaccination and Treatment 
for SIR model [refer].

Optimal control theory is applied to suggest the most effective mitigation 
strategy to minimize the number of individuals who become infected in the 
course of an infection while efficiently balancing vaccination and treatment 
applied to the models with various cost scenarios. Optimal control is a  
mathematical technique derived from the calculus of variations.

Accidental and intentional introduction of infectious diseases  to previously 
naive geographic regions has brought more focus and attention to the 
development of response plans to such scenarios.

All levels of government and public health officials are searching for answers 
to identify the best strategies for intervention prior to what may be an 
inevitable event.

The first step is to represent the epidemiology of the disease divided in 
subpopulations, called compartments, that represent the various stage of the 
disease progression.

For example, Pontryagin’s maximum principle allows the calculation of the 
optimal control for an ordinary differential equation model system with a given 
constraint.

The question  we study in this case is whether this underlying epidemic 
structure significantly impacts the predicted optimal control strategy for 
administering vaccination and / or treatment.

We present the SIR model analysis in detail including, the proof of uniqueness 
and existence of the optimal control system. In this model, we consider a 
logistic growth as defined by the following state equation:

\begin{eqnarray}\label{eq.1.4.SIR}
\dot{S}  &=& \mu N-\beta\frac{SI}{N}-\nu S -\mu \frac{NS}{K}, \label{eq.1.4.1}  
\\
\dot{I}   &=& \beta\frac{SI}{N} - (\gamma+\tau+\delta) I -\mu \frac{IN}{K}, 
\label{eq.1.4.2} \\
\dot{R} &=& (\gamma+\tau) I+\nu S - \mu \frac{NR}{K},  \label{eq.1.4.3}
\end{eqnarray}

subject to the boundary conditions

\begin{equation}\label{eq.1.4.4}
S(0)=S_0,\,\, I(0)=I_0,\,\, R(0)=R_0.
\end{equation}

The variables,  $S,I, R$ represent the susceptible, infections and removed 
classes, respectively. We assume a logistic growth of the total population 
$N=S+I+R$ with carrying capacity $K$, and assume the new births enter in the 
susceptible class.

$\beta :$ approximates the average number of contacts with infectious 
individuals needed to make one person ill in each unit of time.
$\mu :$ approximates the average number of individuals that birth/death in each 
unit of time.

$\gamma :$ approximates the average number of infectious individuals removed 
from infection class to removed class in each unit of time.

$\delta :$ approximates the average number of individuals that dies by the 
disease in each unit of time.

We see that $\beta \frac{SI}{N}$ is the number of susceptibles becomes 
infectious in each unit of time, $\mu \frac{SN}{K}$ is the number of 
susceptibles that dies in each unit of time, similarly in the other terms. 

$\nu (t):$ control function measures the rates at which susceptibles are 
vaccinated in each time period.

$\tau (t):$ control function measures the rates at which infectious individuals 
are treated in each time period.

Note that control $\nu$ moves individuals from the $S$ class to the $R$ class 
and the control $\tau$ moves individuals from the $I$ class to the $R$ class.

Given initial population sizes $S_0,I_0,R_0$, we seek the best mitigation 
strategy for the outbreak modeled in equation (\ref{})-(\ref{}) by optimally 
defining bounded, Lebesgue integrable control functions $\nu (t)$ and $\tau 
(t)$.

Our goal is to minimize the number of people who become infected, and thus also 
the number of people who die due to the infection, while also minimizing the 
effort of vaccinating and treating the population. Thus we seek minimizing the 
following objective functional

\begin{equation}\label{eq.1.4.5}
J(\nu,\tau)=\int_{0}^{T} [B_1 I(t)+B_2[\frac{R(t)}{K}]^m \nu^2(t)+B_3\tau^2(t)] 
dt,
\end{equation}

where $m \geq 1$.

The constants $B_1,B_2,B_3$ have a dual role. On one hand, they are needed to 
balance the units in the integrand and $\nu$ and $\tau$ are treatment rates and 
will be necessarily lies between 0 and 1.

The remaining term in our objective functional seeks to increase the expense of 
vaccination when most of the population has either been vaccinated or has 
immunity from a prior infection.

We assume there are practical limitations on the maximum rate at which 
individuals who may be vaccinated or treated in a given time period and we 
define positive constants $\nu_{max}$ and $\tau_{max}$ accordingly.
We define the set of admissible controls to be 

\begin{equation}\label{eq.1.4.6}
\Omega =\{(\nu,\tau)\in L^1(0,T) |  (\nu (t), \tau (t))\in [0,\nu_{max}]\times 
[0,\tau_{max}], \mbox{for every}\, t\in [0,T]\}.
\end{equation}

We seek an optimal control pair $(\nu^*,\tau^*)$ such that

\begin{equation}\label{eq.1.4.7}
J(\nu^*,\tau^*)=\min_{\Omega}J(\nu,\tau).
\end{equation}

Now, we need deduce the sufficient conditions for the existence of a solution 
to the optimal control problem.

A high death rate from disease could theoretically cause the population to 
vanish, and the existence theorem must assume that the external death rate 
$\delta$ is smaller that the population intrinsic growth, or turnover, rate 
$\mu$.

\begin{theorem}\label{Teo.Ex.Ap}
There exists an optimal control pair $\nu^*(t),\tau^*(t),$ and corresponding 
solution $S^*,I^*,R^*$ to the state initial value problem 
(\ref{eq.1.4.1})-(\ref{eq.1.4.4}) that minimizes $J(\nu,\tau)$ over $\Omega$.
\end{theorem}

\begin{proof}
We referer to the conditions in Theorem III.4.1 and its Corollary in Flaming 
and Rishel []. The requirements there on the set of admissible controls 
$\Omega$ and on the set of end conditions are clearly met here.

The following nontrivial requirements from Fleming and Rishel’s theorem are 
listed and later verified below:

The set of all solutions to system (1)-(4) with corresponding control functions 
in $\Omega$ is nonempty.
The state system can be written as a linear function of the control variables 
with coefficients dependent on time and the state variables.
The integrand $L$ in equation (\ref{eq.1.4.6}) is convex on $\Omega$ and 
additionally satisfies $L(t,S,I,R,\nu,\tau) \geq c_1 |(\nu,\tau)|^{\beta}-c_2$, 
where $c_1>0$ and $\beta>1$.
In order to establish condition a), we refer to Theorem 3.1, by 
Picard-Lindelof, (existence theorem of solution for EDO). If the solutions to 
the state equations are a priori bounded and if the state equations are 
continuous and Lipschitz in the state variables, then there is a unique 
solution corresponding to every admissible control pair in $\Omega$.

Thus, we begin establishing bounds on $N=S+I+R$, and, by extension, $S,I,R$. 
Note that N satisfies the modified logistic equation

\begin{equation}\label{eq.1.4.8}
\dot{N}=\mu N(1-\frac{N}{K})-\delta I,
\end{equation}

and assume that $N(0)=N_0$. We can see that, if $N>K$, then N is decreasing. 
Thus, $N$ is bounded about by $N_0$. If $N<K$, then N is increasing. Thus, 
again $N$ is bounded by $K$. This imply that $N$ is bounded by $\max (N_0,K)$ 
and $\min (N_0,K)$. Note that by extension, we can see that $S,I,R$ are upper 
bounded by the same bound for $N$.

With bounds established above, its follows that the state system is continuous 
and bounded. it is equally direct to show the boundedness of the partial 
derivates with respect to the state variables in the state system, which 
establishes that the system is Lipschitz with respect to the state variables. 
This completes the proof that condition a) holds.

Condition b) is verified by observing the linear dependence of the state 
equations on controls $\nu$ and  $\tau$.

Finally, to verify condition c) we note that since the integrand $L$ of the 
objective functional is quadratic in the controls. $L$ is convex in the 
controls. To prove that bound on the $L$ we note that by the definition of 
$\Omega$ we have $B_2 \nu^2\leq B_2$, and $B_2\nu^2-B_2\leq 0$. Thus $L=B_1 
I(t)+B_2[\frac{R}{K}]^m \nu^2(t) +B_3\tau^2(t)\geq B_3\tau^2 (t)\geq 
B_2\nu^2+B_3\tau^2-B_2\geq \min (B_2,B_3)(\nu^2+\tau^2)-B_2$.
\end{proof}

Now, we apply Pontryagin's Maximun Principle and convert the optimization 
problem to the problem of finding the point-wise minimun relative to $\nu$ and 
$\tau$ of the Hamiltonian.

\begin{theorem}
Given the optimal control pair $(\nu^*,\tau^*)$ and corresponding solutions to 
the state system (\ref{eq.1.4.1})-(\ref{eq.1.4.4}) $S^*,I^*,R^*$ that minimize 
the objetive functional (\ref{eq.1.4.5}) there exist adjoint variables 
$\lambda_1$, $\lambda_2$ and $\lambda_3$ satisying

\begin{eqnarray}\label{eq.1.4.adj}
\dot{\lambda_1}&=& \frac{\mu}{K}[\lambda_1(S+N-K)+\lambda_2 I+\lambda_3 
R]+\beta \frac{I(N-S)}{N^2}(\lambda_1-\lambda_2)+\nu(\lambda_1-\lambda_3), 
\label{eq.1.4.10}\\
\dot{\lambda_2}&=& -B_1+\frac{\mu}{K}[\lambda_1(S-K)+\lambda_2(N+I)+\lambda_3 
R]+\beta\frac{S(N-I)}{N^2}(\lambda_1-\lambda_2)\nonumber\\
&+&\delta \lambda_2+(\gamma+\tau)(\lambda_2-\lambda_3),\label{eq.1.4.11}\\
\dot{\lambda_3}&=& mB_2 \nu^2\frac{R^{m-1}}{K^m}+\frac{\mu}{K}[(S-K)\lambda_1 
+\lambda_2 I+(N+R)\lambda_3]+\beta 
\frac{SI}{N^2}(\lambda_2-\lambda_1),\label{eq.1.4.12}
\end{eqnarray}

with transversality conditions

\begin{equation}\label{eq.1.4.13}
\lambda_1(T)=\lambda_2(T)=\lambda_3(T)=0.
\end{equation}

Fuethermore, as long as the optimal removed class $R^*$ is nonzero, we may 
characterize the optimal pair by the continuous functions

\begin{eqnarray}
\nu^*&=&\min 
\left(\max\left(0,\frac{S^*(\lambda_1-\lambda_3)}{2B_2[\frac{R^*}{K}]^m}\right),\nu_{max}\right),\nonumber\\
\tau^*&=&\min 
\left(\max\left(0,\frac{I^*(\lambda_2-\lambda_3)}{2B_3}\right),\tau_{max}\right)
 \label{eq.1.4.14}.
\end{eqnarray}
\end{theorem}

\begin{proof}
The result follows from a direct application o a version of Pontryagin's 
Maximum Principle for bounded controls [Morton]. We form the Hamiltonian H:

$$H=g(t,u(t),X(t))+\lambda_1\dot{S}+\lambda_2\dot{I}+\lambda_3\dot{R}.$$
Thus,

\begin{eqnarray}\label{eq.1.4.15}
H&=&B_1I+B_2 [\frac{R}{K}]^m\nu^2+B_3\tau^2+\lambda_1[\nu 
N-\mu\frac{N}{K}S-\beta\frac{SI}{N}-\nu S]\nonumber\\
&+&\lambda_2[\beta 
\frac{SI}{N}-(\delta+\gamma+\tau)I-\mu\frac{NI}{K}]+\lambda_3[(\gamma+\tau)I+\nu
 S-\mu\frac{NR}{K}].
\end{eqnarray}

Again, by the Pontryagin's Maximum Principle, The adjoint equations are given 
by the equations $\dot{\lambda_1}=-\frac{\partial H}{\partial S}$, 
$\dot{\lambda_2}=-\frac{\partial H}{\partial I}$, 
$\dot{\lambda_3}=-\frac{\partial H}{\partial R}$, and must satisfy 
transversality conditions $\lambda_i(T)=0$ for $i=1,2,3$. 

Finally, the optimal conditions dictate that $\frac{\partial H}{\partial 
\nu}=\frac{\partial H}{\partial \nu}=0$, then

$$\frac{\partial H}{\partial \nu}=2B_2\left[\frac{R}{K}\right]^m\nu-\lambda_1 
S+\lambda_3 S\left|_{(S^*,I^*,R^*)}\right.=0,$$

and we get 

$$\nu^*=\frac{S^*(\lambda_1-\lambda_3)}{2B_2\left[\frac{R}{K}\right]^m}.$$

$$\frac{\partial H}{\partial \tau}=2B_3\tau-\lambda_2 I+\lambda_3 
I\left|_{(S^*,I^*,R^*)}\right.=0,$$

and we get 

$$\tau^*=\frac{I^*(\lambda_2 -\lambda_3)}{2B_3}.$$

This values are not necessay positive, then we need $\max\left 
(0,\frac{S^*(\lambda_1-\lambda_3)}{2B_2\left[\frac{R}{K}\right]^m}\right )$, 
$\max\left ( 0,\frac{I^*(\lambda_2 -\lambda_3)}{2B_3}\right )$, and by 
boundness we get (\ref{eq.1.4.14}). Observe that by the characterization of the 
controls given in (\ref{eq.1.4.14}) and the nonzero assumption for $R$, it 
follows that the controls are continuous in time.	
\end{proof}

The optimality system is defined as is the compilation of the state equations 
(\ref{eq.1.4.1})-(\ref{eq.1.4.3}), the initial conditions (\ref{eq.1.4.4}), the 
adjoint equations (\ref{eq.1.4.10})-(\ref{eq.1.4.12}), and the transversality 
conditions (\ref{eq.1.4.13}), with the optimality equations (\ref{eq.1.4.14}) 
sustituted into the state and adjoint equations.

In the previous sections we present results or which we guarantee the existence 
of an optimal pair, but the uniqueness off this pair can not be guaranteed. 
Therefore, we present, for this system, the following result in which we 
guarantee the uniqueness of the optimal pair for a small time interval.

\begin{theorem}
	For $T$ sufficiently small the optimality system is unique.
\end{theorem}
\begin{proof}
We must first consider bounds on the adjoint system. Note that bounds for the 
state system were established in Theorem	\ref{Teo.Ex.Ap}. To see that the 
adjoint system is bounded, we rearrange equations 
(\ref{eq.1.4.10})-(\ref{eq.1.4.12}):

\begin{eqnarray*}
\dot{\lambda_1}&=& 
\lambda_1\left[\frac{\mu}{K}(S+N-K)+\beta\frac{I(N-S)}{N^2}+\nu\right]+\lambda_2\left[\frac{\mu}{K}I-\beta\frac{I(N-S)}{N^2}\right]+\lambda_3\left[\frac{\mu}{K}R-\nu\right],\\
\dot{\lambda_2}&=& 
-B_1+\lambda_1\left[\frac{\mu}{K}(S-K)+\beta\frac{S(N-I)}{N^2}\right]+\lambda_2\left[\frac{\mu}{K}(N+I)-\beta\frac{S(N-I)}{N^2}+\delta+\gamma+\tau
 \right]\\
&+&\lambda_3\left[\frac{\mu}{K}R-\gamma-\tau\right],\\
\dot{\lambda_3}&=& 
mB_2\nu^2\frac{R^{m-1}}{K^m}+\lambda_1\left[\frac{\mu}{K}(S-K)-\beta\frac{SI}{N^2}\right]+\lambda_2\left[\frac{\mu}{K}I+\beta\frac{SI}{N^2}\right]+\lambda_3\left[\frac{\mu}{K}(N+R)\right].
\end{eqnarray*}

We recall that $N=S+I+R$, that $S,I,R$ are nonnegative, and that $N>0$. Thus, 
$0\leq \frac{I(N-S)}{N^2},\frac{SI}{N^2},\frac{S(N-I)}{N^2}\leq 1$. We recall 
the bounds $0\leq \nu\leq \nu_{max}$ and $0\leq \tau \leq \tau_{max}$, and with 
thid substitution we see the following bounds for each equation:

\begin{eqnarray*}
\dot{\lambda_1}&=& 
\lambda_1\left[\frac{\mu}{K}(S+N-K)+\beta\frac{I(N-S)}{N^2}+\nu\right]+\lambda_2\left[\frac{\mu}{K}I-\beta\frac{I(N-S)}{N^2}\right]+\lambda_3\left[\frac{\mu}{K}R-\nu\right]\\
&\leq& 
\lambda_1\left[\frac{\mu}{K}(S+N-K)+\beta+\nu_{max}\right]+\lambda_2\left[\frac{\mu}{K}I-\beta\right]+\lambda_3\left[\frac{\mu}{K}R-\nu_{max}\right].
\end{eqnarray*}

\begin{eqnarray*}
	\dot{\lambda_2}&=& 
	-B_1+\lambda_1\left[\frac{\mu}{K}(S-K)+\beta\frac{S(N-I)}{N^2}\right]+\lambda_2\left[\frac{\mu}{K}(N+I)-\beta\frac{S(N-I)}{N^2}+\delta+\gamma+\tau
	 \right]\\
	&+&\lambda_3\left[\frac{\mu}{K}R-\gamma-\tau\right]\leq-B_1+\lambda_1\left[\frac{\mu}{K}(S-K)+\beta\right]+\lambda_2\left[\frac{\mu}{K}(N+I)-\beta+\delta+\gamma+\tau_{max}
	 \right]\\
	&+&\lambda_3\left[\frac{\mu}{K}R-\gamma-\tau_{max}\right].
\end{eqnarray*}
	
\begin{eqnarray*}
	\dot{\lambda_3}&=& 
	mB_2\nu^2\frac{R^{m-1}}{K^m}+\lambda_1\left[\frac{\mu}{K}(S-K)-\beta\frac{SI}{N^2}\right]+\lambda_2\left[\frac{\mu}{K}I+\beta\frac{SI}{N^2}\right]+\lambda_3\left[\frac{\mu}{K}(N+R)\right]\\
	&\leq& 
	mB_2\nu^2_{max}+\lambda_1\left[\frac{\mu}{K}(S-K)-\beta\right]+\lambda_2\left[\frac{\mu}{K}I+\beta\right]+\lambda_3\left[\frac{\mu}{K}(N+R)\right]
\end{eqnarray*}

Hence, the adjoint system is bounded by linear systems with bounded 
coefficients. Thus, the sub- and super-solutions are uniformly bounded, 
establishing bounds for the adjoint system in finite time.

Now, suppose that there are two solutionsto the optimality system:

$$(S(t),I(t),R(t),\lambda_1(t),\lambda_2(t),\lambda_3(t))\,\,\mbox{and}\,\, 
(\bar{S}(t),\bar{I}(t),\bar{R}(t),\bar{\lambda}_1(t),\bar{\lambda}_2(t),\bar{\lambda}_3(t)).$$

To show that the two solutions are equivalent, it is convenient to make a 
change of variables. We define 
$s,i,r,\phi_1,\phi_2,\phi_3,\bar{s},\bar{i},\bar{r},\bar{\phi}_1,\bar{\phi}_2$ 
and $\bar{\phi}_3$ so that

$$S(t)=e^{\alpha t}s(t),\, I(t)=e^{\alpha t}i(t)\, R(t)=e^{\alpha t}r(t),$$

$$\lambda_1(t)=e^{-\alpha t}\phi_1(t), \lambda_2(t)=e^{-\alpha t}\phi_2(t),\, 
\lambda_3(t)=e^{-\alpha t}\phi_3(t),$$

where $\alpha$ is a constant to be chosen later. The barred variables are 
transformed similarly. Note that the bounds for the state and adjoint variables 
can be extended to bounds for the new variables. With the new variables the 
optimality conditions become

\begin{equation}\label{eq.1.4.33}
\nu=\min\left(\max\left(0,\frac{e^{2\alpha 
t}s(t)(\phi_1(t)-\phi_3(t))}{2B_2\left[\frac{e^{\alpha 
t}r(t)}{K}\right]^m}\right),\nu_{max}\right),
\end{equation}

\begin{equation}\label{eq.1.4.34}
\tau=\min\left(\max\left(0,\frac{e^{2\alpha 
t}i(t)(\phi_2(t)-\phi_3(t))}{2B_3}\right),\tau_{max}\right),
\end{equation}

and likewise the optimality conditions for the barred equations could be 
defined. For convenience we define $n(t)=s(t)+i(t)+r(t)$ and we note that 
$N(t)=e^{\alpha t}n(t)$. Note that we omit the dependence on time in the 
following except in the case that a specific time is intended. In the next 
part, we consider the difference of the resulting equations for $s$ and 
$\bar{s}$, $i$ and $\bar{i}$, and so on, and we the simpliy the resulting 
equations by integration with the use of appropiate integrating factors.

\begin{eqnarray*}
	\alpha e^{\alpha t}s+ e^{\alpha t}\dot{s}&=&\mu e^{\alpha t}n-\beta 
	\frac{e^{2\alpha t}si}{e^{\alpha t} n}-\nu e^{\alpha t} s-\mu 
	\frac{e^{2\alpha t} ns}{K},\\
	\alpha s+\dot{s}&=&\mu n-\beta \frac{si}{n}-\nu s-\mu e^{\alpha t}\frac{n 
	s}{k}.
\end{eqnarray*}

Now, we substract from the above the corresponding barred equation to get:

\begin{equation*}
\alpha 
(s-\bar{s})+(\dot{s}-\dot{\bar{s}})=\mu(n-\bar{n})-\beta\left[\frac{si}{n}-\frac{\bar{s}\bar{i}}{\bar{n}}\right]-(\nu
 s-\bar{\nu}\bar{s})-\mu e^{\alpha 
t}\left[\frac{sn}{K}-\frac{\bar{s}\bar{n}}{K}\right]. 
\end{equation*}

We multiply by $(s-\bar{s})$ and integrate from 0 (at which  the state equation 
variables are equivalent) to $T$ yielding

\begin{eqnarray}
\frac{1}{2}(s(T)-\bar{s}(T))^2&+&\alpha\int_{0}^{T}(s-\bar{s})^2dt=\mu\int_{0}^{T}(n-\bar{n})(s-\bar{s})dt-\beta\int_{0}^{T}\left[\frac{si}{n}-\frac{\bar{s}\bar{i}}{\bar{n}}\right](s-\bar{s})dt\nonumber\\
&-&\int_{0}^{T}(\nu s-\bar{\nu}\bar{s})(s-\bar{s})dt-\frac{\mu}{K}\int_{0}^{T} 
e^{\alpha t}\left[sn-\bar{s}\bar{n}\right](s-\bar{s})dt. \label{eq.1.4.35}
\end{eqnarray}

The remaining equations are manipulated similarly, with derivations and 
dependence on time omitted in the interest of space. The specific 
characterization of the controls as given in equations (\ref{eq.1.4.33}) and 
(\ref{eq.1.4.34}) is represented simply by $\nu,\bar{\nu},\tau$ and 
$\bar{\tau}$ in the six manipulated equations:

\begin{eqnarray}
\frac{1}{2}(i(T)-\bar{i}(T))^2&+&\alpha\int_{0}^{T}(i-\bar{i})^2dt=\beta\int_{0}^{T}\left[\frac{si}{n}-\frac{\bar{s}\bar{i}}{\bar{n}}\right](i-\bar{i})dt-(\gamma+\delta)\int_{0}^{T}(i-\bar{i})^2dt\nonumber\\
&-&\int_{0}^{T}(\tau 
i-\bar{\tau}\bar{i})(i-\bar{i})dt-\frac{\mu}{K}\int_{0}^{T} e^{\alpha 
t}\left[in-\bar{i}\bar{n}\right](i-\bar{i})dt. \label{eq.1.4.36}
\end{eqnarray}

\begin{eqnarray}
\frac{1}{2}(r(T)-\bar{r}(T))^2&+&\alpha\int_{0}^{T}(r-\bar{r})^2dt=\gamma\int_{0}^{T}(i-\bar{i})(r-\bar{r})dt+\int_{0}^{T}(\tau
 i-\bar{\tau}\bar{i})(r-\bar{r})dt\nonumber\\
&+&\int_{0}^{T}(\nu s-\bar{\nu}\bar{s})(r-\bar{r})dt-\frac{\mu}{K}\int_{0}^{T} 
e^{\alpha t}\left[rn-\bar{r}\bar{n}\right](r-\bar{r})dt. \label{eq.1.4.37}
\end{eqnarray}

And for adjoint variables we get:

\begin{eqnarray}
-\frac{1}{2}(\phi_1(0)-\bar{\phi_1}(0))^2&-&\alpha\int_{0}^{T}(\phi_1-\bar{\phi_1})^2dt=\frac{\mu}{K}\int_{0}^{T}e^{\alpha
 t}(\phi_1-\bar{\phi}_1)\{[\phi_1(n+s)+\phi_2 i +\phi_3 r]dt\nonumber\\
&-&[\bar{\phi}_1(\bar{n}+\bar{s})+\bar{\phi}_2 \bar{i} +\bar{\phi}_3 
\bar{r}]\}dt-\mu\int_{0}^{T}(\phi_1-\bar{\phi_1})^2dt \nonumber\\
&+&\beta\int_{0}^{T}e^{\alpha 
t}(\phi_1-\bar{\phi}_1)\left[\frac{i(n-s)}{n^2}(\phi_1-\phi_2)-\frac{\bar{i}(\bar{n}-\bar{s})}{\bar{n}^2}(\bar{\phi}_1-\bar{\phi}_2)\right]dt
 \nonumber\\
&+&\int_{0}^{T} 
(\phi_1-\bar{\phi}_1)[\nu(\phi_1-\phi_3)-\bar{\nu}(\bar{\phi_1}-\bar{\phi_3}))]dt.
 \label{eq.1.4.38}
\end{eqnarray}

\begin{eqnarray}
-\frac{1}{2}(\phi_2(0)-\bar{\phi_2}(0))^2&-&\alpha\int_{0}^{T}(\phi_2-\bar{\phi_2})^2dt=\frac{\mu}{K}\int_{0}^{T}e^{\alpha
 t}(\phi_1s-\bar{\phi}_1\bar{s})(\phi_2-\bar{\phi_2})dt\nonumber\\
&-&\mu\int_{0}^{T}(\phi_1-\bar{\phi_1})(\phi_2-\bar{\phi_2})dt+\frac{\mu}{K}\int_{0}^{T}e^{\alpha
 t}[\phi_2(n+i)-\bar{\phi}_2(\bar{n}+\bar{i}) \nonumber\\
&+& (\phi_3 
r-\bar{\phi_3}\bar{r})](\phi_2-\bar{\phi}_2)dt+\beta\int_{0}^{T}\left[\frac{s(n-i)}{n^2}(\phi_1-\phi_2)-\frac{\bar{s}(\bar{n}-\bar{i})}{\bar{n}^2}(\bar{\phi}_1-\bar{\phi}_2)\right]
 \nonumber\\
&+&(\phi_2-\bar{\phi_2})dt+\delta\int_{0}^{T}(\phi_2-\bar{\phi_2})^2dt 
+\gamma\int_{0}^{T}[(\phi_2-\phi_3)-(\bar{\phi}_2-\bar{\phi}_3)](\phi_2-\bar{\phi_2})dt
 \nonumber\\
&+&\int_{0}^{T}[\tau(\phi_2-\phi_3)-\bar{\tau}(\bar{\phi}_2-\bar{\phi}_3)](\phi_2-\bar{\phi_2})dt.
 \label{eq.1.4.39}
\end{eqnarray}

\begin{eqnarray}
-\frac{1}{2}(\phi_3(0)-\bar{\phi}_3(0))^2&-&\alpha\int_{0}^{T}(\phi_3-\bar{\phi}_3)^2dt=\frac{m
 B_2}{K^m}\int_{0}^{T}e^{m\alpha 
t}[\nu^2r^{m-1}-\bar{\nu}^2\bar{r}^{m-1}](\phi_3-\bar{\phi}_3)dt\nonumber\\
&+&\frac{\mu}{K}\int_{0}^{T}e^{\alpha t}(\phi_1 
s-\bar{\phi_1}\bar{s})(\phi_3-\bar{\phi_3})dt 
-\mu\int_{0}^{T}(\phi_1-\bar{\phi_1})(\phi_3-\bar{\phi_3})dt\nonumber\\
&+&\beta\int_{0}^{T}\left[\frac{s i}{n^2}(\phi_2-\phi_1)-\frac{\bar{s} 
\bar{i}}{\bar{n}^2}(\bar{\phi}_2-\bar{\phi}_1)\right](\phi_3-\bar{\phi_3})dt+\frac{\mu}{K}\int_{0}^{T}e^{\alpha
 t}\{[ \nonumber\\
&&\phi_2 i+\phi_3(n+r)]-[\bar{\phi}_2 
\bar{i}+\bar{\phi}_3(\bar{n}+\bar{r})]\}(\phi_3-\bar{\phi_3})dt. 
\label{eq.1.4.40}
\end{eqnarray}

We define

\begin{equation*}
	\Psi(t)=(s(t)-\bar{s}(t))^2+(i(t)-\bar{i}(t))^2+(r(t)-\bar{r}(t))^2
\end{equation*}

and

\begin{equation*}
\Phi(t)=(\phi_1(t)-\bar{\phi_1}(t))^2+(\phi_2(t)-\bar{\phi_2}(t))^2+(\phi_3(t)-\bar{\phi_3}(t))^2.
\end{equation*}

Observe that $\Psi(t)\geq 0$ and $\Phi(t)\geq 0$ for every $t$. We multiply 
equations (\ref{eq.1.4.38})-(\ref{eq.1.4.40}) by $-1$ and then add the 
resulting equations to equations (\ref{eq.1.4.35})-(\ref{eq.1.4.37}). The 
left-hand side of the resulting equation become

\begin{equation*}
	\frac{1}{2}(\Psi(T)+\Phi(0))+\alpha\int_{0}^{T}(\Psi(t)+\Phi(t))dt,
\end{equation*}

and we work now to bound the right-hand side. An elementary inequality

\begin{equation*}
(x-y)^2\geq 0\,\,\mbox{implies}\,\, x^2+y^2\geq \frac{1}{2}(x^2+y^2)\geq xy
\end{equation*}

can be used repeatedly to simplify right-hand expressions. The first is in the 
common form

\begin{equation}\label{eq.1.4.41}
(x-\bar{x})(y-\bar{y})\leq (x-\bar{x})^2+(y-\bar{y})^2.
\end{equation}

Another common expression and needed inequality:

\begin{eqnarray*}
(xy-\bar{x}\bar{y})(w-\bar{w})&=& 
(xy-\bar{x}y+\bar{x}y-\bar{x}\bar{y})(w-\bar{w})\\
&=& y(x-\bar{x})(w-\bar{w})+\bar{x}(y-\bar{y})(w-\bar{w})\\
&\leq&y[(x-\bar{x})^2+(w-\bar{w})^2]+\bar{x}[(y-\bar{y})^2+(w-\bar{w})^2]\\
&=& [y(x-\bar{x})^2+\bar{x}(y-\bar{y})^2+(y+\bar{x})(w-\bar{w})^2].
\end{eqnarray*}

Let define $C=(y+\bar{x})$, where $C$ depends on bounds for $\bar{x}$ and $y$, 
to get

\begin{equation}\label{eq.1.4.42}
(xy-\bar{x}\bar{y})(w-\bar{w})\leq C[(x-\bar{x})^2+(y-\bar{y})^2+(w-\bar{w})^2].
\end{equation}

Several bounds require in turn a bound on $(n-\bar{n})^2$ which follows 
directly from the definition of $n$:

\begin{equation}\label{eq.1.4.43}
(n-\bar{n})^2\leq (s-\bar{s})^2+(i-\bar{i})^2+(r-\bar{r})^2.
\end{equation}

A bound is needed for the expressions with division by $n^2$ in equations 
(\ref{eq.1.4.38})-(\ref{eq.1.4.40}). We focus on the árticular expression in 
equation (\ref{eq.1.4.40}), and others would be similar. Note that

\begin{equation*}
\frac{si}{n^2}-\frac{\bar{s}\bar{i}}{\bar{n}^2}=\frac{1}{n^2\bar{n}^2}[(i-\bar{i})s\bar{n}^2+\bar{i}\bar{n}^2(s-\bar{s})+\bar{i}\bar{s}(\bar{n}-n)(\bar{n}+n)].
\end{equation*}

In the following, for simplicity we write $Q=\frac{si}{n^2}$ with similar 
definition for $\bar{Q}$, and rely on bounds for the state variables 
established in Theorem \ref{Teo.Ex.Ap}.

\begin{eqnarray}
\left[\frac{si}{n^2}(\phi_2-\phi_1)-\frac{\bar{s}\bar{i}}{\bar{n}^2}(\bar{\phi}_1-\bar{\phi_2})\right]&=&\{(\phi_2-\phi_1)\frac{1}{n^2\bar{n}^2}[(i-\bar{i})s\bar{n}^2+\bar{i}\bar{n}^2(s-\bar{s})\nonumber\\
&+&\bar{i}\bar{s}(\bar{n}-n)(\bar{n}+n)]+Q(\phi_2-\bar{\phi_2})\nonumber\\
&+&\left. Q(\bar{\phi_1}-\phi_1)\right\}(\phi_3-\bar{\phi_3})\nonumber \\
&\leq& C(\Psi+\Phi). \label{eq.1.4.44}
\end{eqnarray}

where $C$ depends on the bounds for $n$, and by extension $s,i$ and $r$, and 
the bounds for $\phi_1$ and $\phi_2$. Equations (\ref{eq.1.4.35}) and 
(\ref{eq.1.4.36}) have terms containing $\frac{si}{n}$ and can be bounded 
similarly. Using the equality

\begin{equation*}
\frac{si}{n}-\frac{\bar{s}\bar{i}}{\bar{n}}=\frac{s}{n}(i-\bar{i})+\frac{s\bar{i}}{n\bar{n}}(\bar{n}-n)+(s-\bar{s})\frac{\bar{i}}{\bar{n}}
\end{equation*}

the following bound can be shown

\begin{equation}\label{eq.1.4.45}
[(i-\bar{i})-(s-\bar{s})]\left[\frac{si}{n}-\frac{\bar{s}\bar{i}}{\bar{n}}\right]\leq
 C((i-\bar{i})^2+(n-\bar{n})^2+(s-\bar{s})^2),
\end{equation}

with $C$ depends on bounds for $s,i,$ and $r$.

Now, we note that a bound for $(\nu-\bar{\nu})^2$ requieres the equation [cite 
equation], that implies

\begin{eqnarray*}
	[s\bar{r}^m(\phi_1-\phi_3)-\bar{s}r^m(\bar{\phi_1}-\bar{\phi_3})]^2&=&[s\bar{r}^m(\phi_1-\bar{\phi_1})+(s\bar{r}^m-\bar{s}\bar{r}^m+\bar{s}\bar{r}^m-\bar{s}r^m)\bar{\phi_1}\\
	&+&s\bar{r}^m(-\phi_3+\bar{\phi_3})+\bar{\phi_3}(\bar{s}r^m-s\bar{r}^m)]^2\\
	&\leq&11[s^2\bar{r}^{2m}(\phi_1-\bar{\phi_1})^2+s^2\bar{r}^{2m}(\bar{\phi_3}-\phi_3)^2\\
	&+&\bar{r}^{2m}(s-\bar{s})^2(\bar{\phi_1}^2+\bar{\phi_3}^2)+\bar{s}^2(\bar{r}-r)^2(\bar{\phi_1}^2\\
	&+&\bar{\phi_3}^2)(m \max(r,\bar{r})^{m-1})^2].
\end{eqnarray*}	
	
The expression $(\nu-\bar{\nu})^2$ can be approximated following [cite,cite] 
obtaining

\begin{eqnarray}
	(\nu-\bar{\nu})^2&\leq&\left(\frac{e^{2\alpha 
	t}s(\phi_1-\phi_3)}{\frac{2B_2}{K^m}e^{m\alpha t}r^m}-\frac{e^{2\alpha 
	t}\bar{s}(\bar{\phi_1}-\bar{\phi_3})}{\frac{2B_2}{K^m}e^{m\alpha 
	t}\bar{r}^m}\right)^2\nonumber\\
	&\leq& 11\left[\frac{K^me^{(2-m)\alpha t}e^{m\alpha 
	t}}{2B_2R^m}\right]^2[s^2(\phi_1-\bar{\phi_1})^2+s^2(\bar{\phi_3}-\phi_3)^2+(s-\bar{s})^2(\bar{\phi_1}^2+\bar{\phi_3}^2)]\nonumber\\
	&+&11m^2\left[\frac{K^me^{(2-m)\alpha t}e^{m\alpha t}e^{\alpha 
	t}}{2B_2\min(R,\bar{R})^m\max(R,\bar{R})}\right][\bar{s}^2(\bar{r}-r)^2(\bar{\phi_1}^2+\bar{\phi_3}^2)]\nonumber\\
	&\leq&11\left[\frac{K^me^{2\alpha 
	t}}{2B_2}\right]^2[\max(s)^2[(\phi_1-\bar{\phi_1})^2+(\bar{\phi_3}-\phi_3)^2]+(s-\bar{s})^2\max(\phi_1,\phi_3)^2]\nonumber\\
	&+& 11m^2\left[\frac{K^me^{3\alpha 
	t}}{2B_2}\right]^2[\max(s)^24\max(\phi_1,\phi_3)^2(\bar{r}-r)^2]\nonumber\\
	&\leq& Ce^{6\alpha 
	t}[(\phi_1-\bar{\phi_1})^2+(\bar{\phi_3}-\phi_3)^2+(s-\bar{s})^2+(\bar{r}-r)^2]\label{eq.1.4.46}
\end{eqnarray}

with $C$ depends on bounds for $s,\phi_1$ and $\phi_3$ and that $R>1$ which was 
assumed previously in the derivation of necessary conditions.

A bound on the second control depends on bounds for $\phi_2$ and $i$:

\begin{eqnarray}
(\tau-\bar{\tau})^2&\leq&\left(\frac{e^{2\alpha 
t}}{2B_3}[i(\phi_2-\phi_3)-\bar{i}(\bar{\phi_2}-\bar{\phi_3})]\right)^2\nonumber\\
&\leq& Ce^{4\alpha 
t}[(i-\bar{i})^2+(\phi_2-\bar{\phi_2})^2+(\phi_3-\bar{\phi_3})^2+(i-\bar{i})^2].\label{eq.1.4.47}
\end{eqnarray}

The only form that remains to be considered is from equation (\ref{eq.1.4.40}):

\begin{eqnarray}
&&\frac{mB_2}{K^m}\int_{0}^{T}e^{\alpha m 
t}[\nu^2r^{m-1}-\bar{\nu}^2\bar{r}^{m-1}](\phi_3-\bar{\phi_3})dt\nonumber\\ 
&\leq& \frac{mB_2}{K^m}e^{m\alpha 
T}\int_{0}^{T}[\bar{\nu}^2(\bar{r}-r)\max(\bar{r},r)^{m-2}+(\bar{\nu}^2-\nu^2)r^{m-1}](\phi_3-\bar{\phi_3})dt\nonumber\\
&\leq& Ce^{2\alpha 
T}\int_{0}^{T}[\bar{\nu}^2(\bar{r}-r)\max(\bar{R},R)^{m-2}+(\bar{\nu}^2-\nu^2)T^{m-1}](\phi_3-\bar{\phi_3})dt\nonumber\\
&\leq& Ce^{2\alpha 
T}\int_{0}^{T}[(\bar{r}-r)^2+(\bar{\nu}^2-\nu^2)^2+(\phi_3-\bar{\phi_3})^2]dt 
\label{eq.1.4.48}
\end{eqnarray}

where $C$ depends on bounds for $\nu,R$ and the parameters $m,B_2,K$. The 
bounds (\ref{eq.1.4.41})-(\ref{eq.1.4.48}) are aplied to the sum of the 
equations (\ref{eq.1.4.35})-(\ref{eq.1.4.48}), with the latter three modified 
as described previously, deriving an inequality of the form:



with the constants dependent on the parameter values and the established bounds 
for the state and adjoint variables. Rearranging this becomes

\begin{equation}\label{eq.1.4.49}
\frac{1}{2}[\Psi(T)+\Phi(0)]+(\alpha-\tilde{C}+\hat{C}e^{6\alpha 
T})\int_{0}^{T}[\Psi(t)+\Phi(t)]dt\leq 0.
\end{equation}

We now choose $\alpha$ so that 

$$\alpha>\tilde{C}+\hat{C}$$

and note that $\frac{\alpha-\tilde{C}}{\hat{C}}>1$. Subsequently choose $T$ so 
that 

$$T<\frac{1}{6\alpha}\ln\left(\frac{\alpha-\tilde{C}}{\hat{C}}\right).$$

Then

$6\alpha T<\ln\left(\frac{\alpha-\tilde{C}}{\hat{C}}\right)$ implies 
$e^{6\alpha T}<\frac{\alpha-\tilde{C}}{\hat{C}}$.

It follows that $\alpha-\tilde{C}+\hat{C}e^{6\alpha T}>0$, so inequality 
(\ref{eq.1.4.49}) can hold if and only if for every $t\in [0,T]$

$$s(t)=\bar{s}(t),i(t)=\bar{i}(t),r(t)=\bar{r}(t),$$
$$\phi_1(t)=\bar{\phi_1}(t),\phi_2(t)=\bar{\phi_2}(t),\phi_3(t)=\bar{\phi_3}(t)$$

this imply

$$S(t)=\bar{S}(t),I(t)=\bar{I}(t),R(t)=\bar{R}(t),$$
$$\lambda_1(t)=\bar{\lambda_1}(t),\lambda_2(t)=\bar{\lambda_2}(t),\lambda_3(t)=\bar{\lambda_3}(t),$$

establising the uniqueness of the optimal control.
\end{proof}

\begin{eqnarray*}
\frac{1}{2}[\Psi(T)&+&\Phi(0)]+\alpha\int_{0}^{T}[\Psi(t)+\Phi(t)]dt=\mu\int_{0}^{T}(n-\bar{n})(s-\bar{s})dt\\
&+&\beta\int_{0}^{T}\left[\frac{si}{n}-\frac{\bar{s}\bar{i}}{\bar{n}}\right](s-\bar{s})dt+\int_{0}^{T}(\nu
 s-\bar{\nu}\bar{s})(s-\bar{s})dt\\
&+&\frac{\mu}{K}\int_{0}^{T} e^{\alpha 
t}\left[sn-\bar{s}\bar{n}\right](s-\bar{s})dt+\beta\int_{0}^{T}\left[\frac{si}{n}-\frac{\bar{s}\bar{i}}{\bar{n}}\right](i-\bar{i})dt\\
&+&(\gamma+\delta)\int_{0}^{T}(i-\bar{i})^2dt+\int_{0}^{T}(\tau 
i-\bar{\tau}\bar{i})(i-\bar{i})dt+\frac{\mu}{K}\int_{0}^{T} e^{\alpha 
t}\left[in-\bar{i}\bar{n}\right](i-\bar{i})dt\\
&+&\gamma\int_{0}^{T}(i-\bar{i})(r-\bar{r})dt+\int_{0}^{T}(\tau 
i-\bar{\tau}\bar{i})(r-\bar{r})dt+\int_{0}^{T}(\nu 
s-\bar{\nu}\bar{s})(r-\bar{r})dt\\
&+&\frac{\mu}{K}\int_{0}^{T} e^{\alpha 
t}\left[rn-\bar{r}\bar{n}\right](r-\bar{r})dt+\frac{\mu}{K}\int_{0}^{T}e^{\alpha
 t}(\phi_1-\bar{\phi}_1)\{[\phi_1(n+s)+\phi_2 i +\phi_3 r]dt\\
&+&[\bar{\phi}_1(\bar{n}+\bar{s})+\bar{\phi}_2 \bar{i} +\bar{\phi}_3 
\bar{r}]\}dt+\mu\int_{0}^{T}(\phi_1-\bar{\phi_1})^2dt\\
&+&\beta\int_{0}^{T}e^{\alpha 
t}(\phi_1-\bar{\phi}_1)\left[\frac{i(n-s)}{n^2}(\phi_1-\phi_2)-\frac{\bar{i}(\bar{n}-\bar{s})}{\bar{n}^2}(\bar{\phi}_1-\bar{\phi}_2)\right]dt\\
&+&\int_{0}^{T} 
(\phi_1-\bar{\phi}_1)[\nu(\phi_1-\phi_3)-\bar{\nu}(\bar{\phi_1}-\bar{\phi_3}))]dt+\frac{\mu}{K}\int_{0}^{T}e^{\alpha
 t}(\phi_1s-\bar{\phi}_1\bar{s})(\phi_2-\bar{\phi_2})dt\\
&+&\mu\int_{0}^{T}(\phi_1-\bar{\phi_1})(\phi_2-\bar{\phi_2})dt+\frac{\mu}{K}\int_{0}^{T}e^{\alpha
 t}[\phi_2(n+i)-\bar{\phi}_2(\bar{n}+\bar{i}) \nonumber\\
&+& (\phi_3 
r-\bar{\phi_3}\bar{r})](\phi_2-\bar{\phi}_2)dt+\beta\int_{0}^{T}\left[\frac{s(n-i)}{n^2}(\phi_1-\phi_2)-\frac{\bar{s}(\bar{n}-\bar{i})}{\bar{n}^2}(\bar{\phi}_1-\bar{\phi}_2)\right]
 \nonumber\\
&+&(\phi_2-\bar{\phi_2})dt+\delta\int_{0}^{T}(\phi_2-\bar{\phi_2})^2dt 
+\gamma\int_{0}^{T}[(\phi_2-\phi_3)-(\bar{\phi}_2-\bar{\phi}_3)](\phi_2-\bar{\phi_2})dt
 \nonumber\\
&+&\int_{0}^{T}[\tau(\phi_2-\phi_3)-\bar{\tau}(\bar{\phi}_2-\bar{\phi}_3)](\phi_2-\bar{\phi_2})dt\\&+&
\frac{m B_2}{K^m}\int_{0}^{T}e^{m\alpha 
t}[\nu^2r^{m-1}-\bar{\nu}^2\bar{r}^{m-1}](\phi_3-\bar{\phi}_3)dt\\
&+&\frac{\mu}{K}\int_{0}^{T}e^{\alpha t}(\phi_1 
s-\bar{\phi_1}\bar{s})(\phi_3-\bar{\phi_3})dt 
+\mu\int_{0}^{T}(\phi_1-\bar{\phi_1})(\phi_3-\bar{\phi_3})dt\nonumber\\
&+&\beta\int_{0}^{T}\left[\frac{s i}{n^2}(\phi_2-\phi_1)-\frac{\bar{s} 
\bar{i}}{\bar{n}^2}(\bar{\phi}_2-\bar{\phi}_1)\right](\phi_3-\bar{\phi_3})dt+\frac{\mu}{K}\int_{0}^{T}e^{\alpha
 t}\{[ \nonumber\\
&&\phi_2 i+\phi_3(n+r)]-[\bar{\phi}_2 
\bar{i}+\bar{\phi}_3(\bar{n}+\bar{r})]\}(\phi_3-\bar{\phi_3})dt. 
\end{eqnarray*}

Now we factorizing and applied the bounds and get

\begin{eqnarray*}
\frac{1}{2}[\Psi(T)&+&\Phi(0)]+\alpha\int_{0}^{T}[\Psi(t)+\Phi(t)]dt\leq \mu 
C\int_{0}^{T}[(\bar{n}-n)^2+(\bar{s}-s)^2]dt\\
&+&\beta 
C\int_{0}^{T}\left[(\bar{i}-i)^2+\left(\frac{si}{n}-\frac{\bar{s}\bar{i}}{\bar{n}}\right)^2\right]dt+\gamma
 C\int_{0}^{T}[(i-\bar{i})^2+(r-\bar{r})^2]dt\\
&+&C\int_{0}^{T}[(\tau-\bar{\tau})^2+(i-\bar{i})^2+(r-\bar{r})^2]dt+C\int_{0}^{T}[(\nu-\bar{\nu})^2+(s-\bar{s})^2+(r-\bar{r})^2]dt\\
&+&\frac{\mu}{K}e^{\alpha T}C\int_{0}^{T}(\phi_1-\bar{\phi_1})^2dt+\beta 
e^{\alpha T}\int_{0}^{T}[(\phi_1-\bar{\phi_1})^2+\Psi(t)+\Phi(t)]dt\\
&+&\int_{0}^{T}[(\phi_1-\bar{\phi_1})^2+(\nu-\bar{\nu})^2+(\phi_1-\bar{\phi_1})^2+(\phi_3-\bar{\phi_3})^2]dt\\
&+&\frac{\mu}{K}e^{\alpha 
T}C\int_{0}^{T}[(\phi_1-\bar{\phi_1})^2+(s-\bar{s})^2+(\phi_2-\bar{\phi_2})^2]dt+\frac{\mu}{K}e^{\alpha
 T}C\int_{0}^{T}(\phi_2-\bar{\phi_2})^2dt\\
&+& \beta 
C\int_{0}^{T}[\Psi(t)+\Phi(t)]dt+\delta\int_{0}^{T}(\phi_2-\bar{\phi_2})^2dt+\gamma
 C\int_{0}^{T}[(\phi_2-\bar{\phi_2})^2+(\phi_3-\bar{\phi_3})^2]dt\\
&+& 
C\int_{0}^{T}[(\tau-\bar{\tau})^2+(\phi_2-\bar{\phi_2})^2+(\phi_3-\bar{\phi_3})^2]dt\\
&+&Ce^{2\alpha 
T}\int_{0}^{T}[(r-\bar{r})^2(\bar{\nu}^2-\nu^2)^2+(\phi_3-\bar{\phi_3})^2]dt\\
&+&\frac{\mu}{K}e^{\alpha 
T}C\int_{0}^{T}[(s-\bar{s})^2+(\phi_1-\bar{\phi_1})^2+(\phi_3-\bar{\phi_3})^2]dt
 + \frac{\mu}{K}e^{\alpha T}C\int_{0}^{T}[(\phi_3-\bar{\phi_3})^2\Psi(t)]dt\\
&+&\beta C\int_{0}^{T}[\Psi (t)+\Phi(t)]dt.
\end{eqnarray*}

The next steps are to apply the bounds on the controls










